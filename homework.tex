\documentclass[UTF8,zihao=-4]{ctexart}
\usepackage{graphicx}
\usepackage{geometry}
\usepackage{siunitx}
\usepackage{amsmath}
\usepackage{amsfonts}
\usepackage{amssymb}
\usepackage{listings}
\usepackage{forest}
\usepackage{tikz}
\lstset{
	language=C++,
	breaklines,
	tabsize=4,
	basicstyle=\ttfamily \small
}
\geometry{a4paper,centering,scale=0.8}
\title{\heiti 计算方法\quad 第一次实验}
\author{PB17000005\quad \CJKfontspec{AR PL UKai CN} 赵作竑}
\date{\kaishu \today}
\begin{document}
	\maketitle
	\paragraph{关于精度}IEEE标准的浮点数,单精度浮点有23位用于表示数值部分,外加省略的1位,一共是24位。$10^7<2^24=16777216<10^8$,也就是说单精度浮点数最多可以表示7位十进制有效数字。因此,问题一只取7位有效数字,而并非题目里面的12位尾数。至于双精度,有53位,$10^{16}<2^53=16777216<10^{17}$,因此双精度浮点数能够表示16位二进制有效数字。
	\paragraph{问题一} 使用C语言编写程序,进行求解。
	\begin{lstlisting}
#include <stdio.h>
#include <math.h>

float f(float x) {
    return sqrtf(x * x + 9) - 3;
}

float g(float x) {
    return x * x / (sqrtf(x * x + 9) + 3);
}

int main() {
    float x[11];
    x[0] = 1;
    printf("x                   ");
    printf("sqrt(x^2+9)-3       ");
    printf("x^2/((sqrt(x^2+9)+3)\n");
    for (int i = 1; i <= 10; ++i) {
        x[i] = x[i - 1] / (float) 8.0;
        printf("%-20e", x[i]);
        printf("%-20e", f(x[i]));
        printf("%-20e\n", g(x[i]));
    }
    return 0;
}
	\end{lstlisting}
	可得计算结果如下表所示:
	\begin{lstlisting}
x                   sqrt(x^2+9)-3       x^2/((sqrt(x^2+9)+3)
1.250000e-01        2.603054e-03        2.603037e-03        
1.562500e-02        4.076958e-05        4.068983e-05        
1.953125e-03        7.152557e-07        6.357828e-07        
2.441406e-04        0.000000e+00        9.934108e-09        
3.051758e-05        0.000000e+00        1.552204e-10        
3.814697e-06        0.000000e+00        2.425319e-12        
4.768372e-07        0.000000e+00        3.789561e-14        
5.960464e-08        0.000000e+00        5.921190e-16        
7.450581e-09        0.000000e+00        9.251859e-18        
9.313226e-10        0.000000e+00        1.445603e-19
	\end{lstlisting}
	从这两种结果来看,我认为第二种方法的结果更可靠,因为当$x\to 0$时,按照$f(x)$计算,$\sqrt{x^2+9}\to 3$,两者相减,有效位数大多用于表示两者相同的位数,而不同的位数已经超出了有效位数的范围,所以很多结果都是0。而第二种方法,当$x\to 0$时,$\sqrt{x^2+9}+3\to 6$,结果更为可靠。
	\paragraph{问题二}用C语言编写程序:
	\begin{lstlisting}
#include <stdio.h>

int main() {
    double arr[5] = {
            4040.045551380452,
            -2759471.276702747,
            -31.64291531266504,
            2755462.874010974,
            0.0000557052996742893
    };
    double sum[3];
    sum[0] = 0;
    for (int i = 0; i < 5; ++i) {
        sum[0] += arr[i];
        sum[1] += arr[4 - i];
    }
    double tmp1 = arr[3], tmp2 = arr[1];
    tmp1 += arr[0];
    tmp1 += arr[4];
    tmp2 += arr[2];
    sum[2] = tmp1 + tmp2;
    printf("             Result\n");
    printf("Method(a)    %-7e\n", sum[0]);
    printf("Method(b)    %-7e\n", sum[1]);
    printf("Method(c)    %-7e\n", sum[2]);

    return 0;
}
	\end{lstlisting}
	可得计算结果如下表所示:
	\begin{lstlisting}
             Result
Method(a)    1.025188e-10
Method(b)    -1.564331e-10
Method(c)    0.000000e+00
	\end{lstlisting}
	那么哪种计算顺序更为精确呢?最精确的计算结果约为$8\times 10^{-11}$,更接近第一种方法的结果。但是,这并不说明第一种方法是一种好方法。
	
	第一种和第二种方法完全按照数字给的顺序相加,本质上是相同,可以看到,如果正向计算,误差很小约为$2\times 10^{-11}$,反向计算,误差很大,约为$2\times 10^{-10}$,相差了一个量级,完全是不可控的,说白了,就是``蒙''而已。
	
	那么第三种方法又如何呢?首先,我们可以看到,结果是一个很小的值,也就是前面的很多位,最后相加都变成了零。如果两个指数相差很大的数字相加,又会导致指数较小的那个数字损失很多有效位数。但是如果大小相近的数字相减,会导致结果的有效数字很少,产生较大的误差,就算指数较小的数字的有效位数没有损失,前面相减产生的误差也使这个失去了意义。两相比较,就要舍弃指数较小的数字的有效数字的位数。至于求和顺序,我认为从大到小不如从小到大的误差小,不过这一步产生的误差在可以接受的范围内。最后一步,两个数字相减,有效位数所剩无几。
\end{document}